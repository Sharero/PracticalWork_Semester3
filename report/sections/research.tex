\section{Тестирование}
Тестирование проводилось на сетках, содержащих 1, 2, 4 и 8 шестигранников. Целью было убедиться в корректности алгоритма разбиения кубических элементов на пирамиды и тетраэдры, а также проверить визуализацию полученной сетки.

\subsection{Тестирование на 1 шестиграннике}
\hspace{-30pt}\textbf{Входные данные:}
\begin{itemize}
	\item \texttt{global\_node\_coordinates:}
	\begin{verbatim}
		0.0 0.0 0.0
		1.0 0.0 0.0
		0.0 1.0 0.0
		1.0 1.0 0.0
		0.0 0.0 1.0
		1.0 0.0 1.0
		0.0 1.0 1.0
		1.0 1.0 1.0
	\end{verbatim}
	\item \texttt{global\_nodes\_indicies:} 	
	\begin{verbatim}
		0 1 2 3 4 5 6 7
	\end{verbatim}
\end{itemize}

\vspace{-24pt}\hspace{-30pt}\textbf{Выходные данные:}
\begin{itemize}
	\item \textbf{Пирамида:} 0(0,0,0) 1(1,0,0) 3(1,1,0) 2(0,1,0) 4(0,0,1)
	\item \textbf{Тетраэдр 1:} 1(1,0,0) 3(1,1,0) 4(0,0,1) 5(1,0,1)
	\item \textbf{Тетраэдр 2:} 3(1,1,0) 2(0,1,0) 4(0,0,1) 6(0,1,1)
	\item \textbf{Тетраэдр 3:} 3(1,1,0) 4(0,0,1) 5(1,0,1) 6(0,1,1)
	\item \textbf{Тетраэдр 4:} 5(1,0,1) 3(1,1,0) 6(0,1,1) 7(1,1,1)
\end{itemize}

\hspace{-30pt}Результат разбиения представлены на рисунке \ref{fig:3}.

\begin{figure}[H]
	\centering
	\includegraphics[width=0.7\textwidth]{./images/1_cube_splitted_grid.png}
	\caption{-- Результат разбиения одного шестигранника}
	\label{fig:3}
\end{figure}

\subsection{Тестирование на 2 шестигранниках}
\hspace{-30pt}\textbf{Входные данные:}
\begin{itemize}
	\item \texttt{global\_node\_coordinates:}
	\begin{verbatim}
		0.0 0.0 0.0
		1.0 0.0 0.0
		2.0 0.0 0.0
		0.0 1.0 0.0
		1.0 1.0 0.0
		2.0 1.0 0.0
		0.0 0.0 1.0
		1.0 0.0 1.0
		2.0 0.0 1.0
		0.0 1.0 1.0
		1.0 1.0 1.0
		2.0 1.0 1.0
	\end{verbatim}
	\item \texttt{global\_nodes\_indicies:} 	
	\begin{verbatim}
		0 1 3 4 6 7 9 10
		1 2 4 5 7 8 10 11
	\end{verbatim}
\end{itemize}

\vspace{-24pt}\hspace{-30pt}\textbf{Выходные данные:}
\begin{itemize}
	\item \textbf{Пирамида 1:} 0(0,0,0) 1(1,0,0) 4(1,1,0) 3(0,1,0) 6(0,0,1) 
	\item \textbf{Пирамида 2:} 1(1,0,0) 2(2,0,0) 5(2,1,0) 4(1,1,0) 7(1,0,1) 
	\item \textbf{Тетраэдр 1:} 1(1,0,0) 4(1,1,0) 6(0,0,1) 7(1,0,1) 
	\item \textbf{Тетраэдр 2:} 4(1,1,0) 3(0,1,0) 6(0,0,1) 9(0,1,1) 
	\item \textbf{Тетраэдр 3:} 4(1,1,0) 6(0,0,1) 7(1,0,1) 9(0,1,1) 
	\item \textbf{Тетраэдр 4:} 7(1,0,1) 4(1,1,0) 9(0,1,1) 10(1,1,1) 
	\item \textbf{Тетраэдр 5:} 2(2,0,0) 5(2,1,0) 7(1,0,1) 8(2,0,1) 
	\item \textbf{Тетраэдр 6:} 5(2,1,0) 4(1,1,0) 7(1,0,1) 10(1,1,1) 
	\item \textbf{Тетраэдр 7:} 5(2,1,0) 7(1,0,1) 8(2,0,1) 10(1,1,1) 
	\item \textbf{Тетраэдр 8:} 8(2,0,1) 5(2,1,0) 10(1,1,1) 11(2,1,1)
\end{itemize}

\hspace{-30pt}Результат разбиения представлены на рисунке \ref{fig:4}.

\begin{figure}[H]
	\centering
	\includegraphics[width=0.7\textwidth]{./images/2_cubes_splitted_grid.png}
	\caption{-- Результат разбиения двух шестигранников}
	\label{fig:4}
\end{figure}

\subsection{Тестирование на 4 шестигранниках}
\hspace{-30pt}\textbf{Входные данные:}
\begin{itemize}
	\item \texttt{global\_node\_coordinates:}
	\begin{verbatim}
		0.0 0.0 0.0
		1.0 0.0 0.0
		2.0 0.0 0.0
		0.0 1.0 0.0
		1.0 1.0 0.0
		2.0 1.0 0.0
		0.0 2.0 0.0
		1.0 2.0 0.0
		2.0 2.0 0.0
		0.0 0.0 1.0
		1.0 0.0 1.0
		2.0 0.0 1.0
		0.0 1.0 1.0
		1.0 1.0 1.0
		2.0 1.0 1.0
		0.0 2.0 1.0
		1.0 2.0 1.0
		2.0 2.0 1.0
	\end{verbatim}
	\item \texttt{global\_nodes\_indicies:} 	
	\begin{verbatim}
		0 1 3 4 9 10 12 13
		1 2 4 5 10 11 13 14
		3 4 6 7 12 13 15 16
		4 5 7 8 13 14 16 17
	\end{verbatim}
\end{itemize}

\vspace{-24pt}\hspace{-30pt}\textbf{Выходные данные:}
\begin{itemize}
	\item \textbf{Пирамида 1:} 0(0,0,0) 1(1,0,0) 4(1,1,0) 3(0,1,0) 9(0,0,1) 
	\item \textbf{Пирамида 2:} 1(1,0,0) 2(2,0,0) 5(2,1,0) 4(1,1,0) 10(1,0,1) 
	\item \textbf{Пирамида 3:} 3(0,1,0) 4(1,1,0) 7(1,2,0) 6(0,2,0) 12(0,1,1) 
	\item \textbf{Пирамида 4:} 4(1,1,0) 5(2,1,0) 8(2,2,0) 7(1,2,0) 13(1,1,1) 
	\item \textbf{Тетраэдр 1:} 1(1,0,0) 4(1,1,0) 9(0,0,1) 10(1,0,1) 
	\item \textbf{Тетраэдр 2:} 4(1,1,0) 3(0,1,0) 9(0,0,1) 12(0,1,1) 
	\item \textbf{Тетраэдр 3:} 4(1,1,0) 9(0,0,1) 10(1,0,1) 12(0,1,1) 
	\item \textbf{Тетраэдр 4:} 10(1,0,1) 4(1,1,0) 12(0,1,1) 13(1,1,1) 
	\item \textbf{Тетраэдр 5:} 2(2,0,0) 5(2,1,0) 10(1,0,1) 11(2,0,1) 
	\item \textbf{Тетраэдр 6:} 5(2,1,0) 4(1,1,0) 10(1,0,1) 13(1,1,1) 
	\item \textbf{Тетраэдр 7:} 5(2,1,0) 10(1,0,1) 11(2,0,1) 13(1,1,1) 
	\item \textbf{Тетраэдр 8:} 11(2,0,1) 5(2,1,0) 13(1,1,1) 14(2,1,1) 
	\item \textbf{Тетраэдр 9:} 4(1,1,0) 7(1,2,0) 12(0,1,1) 13(1,1,1) 
	\item \textbf{Тетраэдр 10:} 7(1,2,0) 6(0,2,0) 12(0,1,1) 15(0,2,1) 
	\item \textbf{Тетраэдр 11:} 7(1,2,0) 12(0,1,1) 13(1,1,1) 15(0,2,1) 
	\item \textbf{Тетраэдр 12:} 13(1,1,1) 7(1,2,0) 15(0,2,1) 16(1,2,1) 
	\item \textbf{Тетраэдр 13:} 5(2,1,0) 8(2,2,0) 13(1,1,1) 14(2,1,1) 
	\item \textbf{Тетраэдр 14:} 8(2,2,0) 7(1,2,0) 13(1,1,1) 16(1,2,1) 
	\item \textbf{Тетраэдр 15:} 8(2,2,0) 13(1,1,1) 14(2,1,1) 16(1,2,1) 
	\item \textbf{Тетраэдр 16:} 14(2,1,1) 8(2,2,0) 16(1,2,1) 17(2,2,1)
\end{itemize}

\hspace{-30pt}Результат разбиения представлены на рисунке \ref{fig:5}.

\begin{figure}[H]
	\centering
	\includegraphics[width=0.7\textwidth]{./images/4_cubes_splitted_grid.png}
	\caption{-- Результат разбиения четырех шестигранников}
	\label{fig:5}
\end{figure}

\subsection{Тестирование на 4 шестигранниках}
\hspace{-30pt}\textbf{Входные данные:}
\begin{itemize}
	\item \texttt{global\_node\_coordinates:}
	\begin{verbatim}
		0.0 0.0 0.0
		1.0 0.0 0.0
		2.0 0.0 0.0
		0.0 1.0 0.0
		1.0 1.0 0.0
		2.0 1.0 0.0
		0.0 2.0 0.0
		1.0 2.0 0.0
		2.0 2.0 0.0
		0.0 0.0 1.0
		1.0 0.0 1.0
		2.0 0.0 1.0
		0.0 1.0 1.0
		1.0 1.0 1.0
		2.0 1.0 1.0
		0.0 2.0 1.0
		1.0 2.0 1.0
		2.0 2.0 1.0
		0.0 0.0 2.0
		1.0 0.0 2.0
		2.0 0.0 2.0
		0.0 1.0 2.0
		1.0 1.0 2.0
		2.0 1.0 2.0
		0.0 2.0 2.0
		1.0 2.0 2.0
		2.0 2.0 2.0
	\end{verbatim}
	\item \texttt{global\_nodes\_indicies:} 	
	\begin{verbatim}
		0 1 3 4 9 10 12 13
		1 2 4 5 10 11 13 14
		3 4 6 7 12 13 15 16
		4 5 7 8 13 14 16 17
		9 10 12 13 18 19 21 22
		10 11 13 14 19 20 22 23
		12 13 15 16 21 22 24 25
		13 14 16 17 22 23 25 26
	\end{verbatim}
\end{itemize}

\vspace{-24pt}\hspace{-30pt}\textbf{Выходные данные:}
\begin{itemize}
	\item \textbf{Пирамида 1:} 0(0,0,0) 1(1,0,0) 4(1,1,0) 3(0,1,0) 9(0,0,1) 
	\item \textbf{Пирамида 2:} 1(1,0,0) 2(2,0,0) 5(2,1,0) 4(1,1,0) 10(1,0,1) 
	\item \textbf{Пирамида 3:} 3(0,1,0) 4(1,1,0) 7(1,2,0) 6(0,2,0) 12(0,1,1) 
	\item \textbf{Пирамида 4:} 4(1,1,0) 5(2,1,0) 8(2,2,0) 7(1,2,0) 13(1,1,1) 
	\item \textbf{Пирамида 5:} 9(0,0,1) 10(1,0,1) 13(1,1,1) 12(0,1,1) 18(0,0,2) 
	\item \textbf{Пирамида 6:} 10(1,0,1) 11(2,0,1) 14(2,1,1) 13(1,1,1) 19(1,0,2) 
	\item \textbf{Пирамида 7:} 12(0,1,1) 13(1,1,1) 16(1,2,1) 15(0,2,1) 21(0,1,2) 
	\item \textbf{Пирамида 8:} 13(1,1,1) 14(2,1,1) 17(2,2,1) 16(1,2,1) 22(1,1,2) 
	\item \textbf{Тетраэдр 1:} 1(1,0,0) 4(1,1,0) 9(0,0,1) 10(1,0,1) 
	\item \textbf{Тетраэдр 2:} 4(1,1,0) 3(0,1,0) 9(0,0,1) 12(0,1,1) 
	\item \textbf{Тетраэдр 3:} 4(1,1,0) 9(0,0,1) 10(1,0,1) 12(0,1,1) 
	\item \textbf{Тетраэдр 4:} 10(1,0,1) 4(1,1,0) 12(0,1,1) 13(1,1,1) 
	\item \textbf{Тетраэдр 5:} 2(2,0,0) 5(2,1,0) 10(1,0,1) 11(2,0,1) 
	\item \textbf{Тетраэдр 6:} 5(2,1,0) 4(1,1,0) 10(1,0,1) 13(1,1,1) 
	\item \textbf{Тетраэдр 7:} 5(2,1,0) 10(1,0,1) 11(2,0,1) 13(1,1,1) 
	\item \textbf{Тетраэдр 8:} 11(2,0,1) 5(2,1,0) 13(1,1,1) 14(2,1,1) 
	\item \textbf{Тетраэдр 9:} 4(1,1,0) 7(1,2,0) 12(0,1,1) 13(1,1,1) 
	\item \textbf{Тетраэдр 10:} 7(1,2,0) 6(0,2,0) 12(0,1,1) 15(0,2,1) 
	\item \textbf{Тетраэдр 11:} 7(1,2,0) 12(0,1,1) 13(1,1,1) 15(0,2,1) 
	\item \textbf{Тетраэдр 12:} 13(1,1,1) 7(1,2,0) 15(0,2,1) 16(1,2,1) 
	\item \textbf{Тетраэдр 13:} 5(2,1,0) 8(2,2,0) 13(1,1,1) 14(2,1,1) 
	\item \textbf{Тетраэдр 14:} 8(2,2,0) 7(1,2,0) 13(1,1,1) 16(1,2,1) 
	\item \textbf{Тетраэдр 15:} 8(2,2,0) 13(1,1,1) 14(2,1,1) 16(1,2,1) 
	\item \textbf{Тетраэдр 16:} 14(2,1,1) 8(2,2,0) 16(1,2,1) 17(2,2,1) 
	\item \textbf{Тетраэдр 17:} 10(1,0,1) 13(1,1,1) 18(0,0,2) 19(1,0,2) 
	\item \textbf{Тетраэдр 18:} 13(1,1,1) 12(0,1,1) 18(0,0,2) 21(0,1,2) 
	\item \textbf{Тетраэдр 19:} 13(1,1,1) 18(0,0,2) 19(1,0,2) 21(0,1,2) 
	\item \textbf{Тетраэдр 20:} 19(1,0,2) 13(1,1,1) 21(0,1,2) 22(1,1,2) 
	\item \textbf{Тетраэдр 21:} 11(2,0,1) 14(2,1,1) 19(1,0,2) 20(2,0,2) 
	\item \textbf{Тетраэдр 22:} 14(2,1,1) 13(1,1,1) 19(1,0,2) 22(1,1,2) 
	\item \textbf{Тетраэдр 23:} 14(2,1,1) 19(1,0,2) 20(2,0,2) 22(1,1,2) 
	\item \textbf{Тетраэдр 24:} 20(2,0,2) 14(2,1,1) 22(1,1,2) 23(2,1,2) 
	\item \textbf{Тетраэдр 25:} 13(1,1,1) 16(1,2,1) 21(0,1,2) 22(1,1,2) 
	\item \textbf{Тетраэдр 26:} 16(1,2,1) 15(0,2,1) 21(0,1,2) 24(0,2,2) 
	\item \textbf{Тетраэдр 27:} 16(1,2,1) 21(0,1,2) 22(1,1,2) 24(0,2,2) 
	\item \textbf{Тетраэдр 28:} 22(1,1,2) 16(1,2,1) 24(0,2,2) 25(1,2,2) 
	\item \textbf{Тетраэдр 29:} 14(2,1,1) 17(2,2,1) 22(1,1,2) 23(2,1,2) 
	\item \textbf{Тетраэдр 30:} 17(2,2,1) 16(1,2,1) 22(1,1,2) 25(1,2,2) 
	\item \textbf{Тетраэдр 31:} 17(2,2,1) 22(1,1,2) 23(2,1,2) 25(1,2,2) 
	\item \textbf{Тетраэдр 32:} 23(2,1,2) 17(2,2,1) 25(1,2,2) 26(2,2,2)
\end{itemize}

\hspace{-30pt}Результат разбиения представлены на рисунке \ref{fig:6}.

\begin{figure}[H]
	\centering
	\includegraphics[width=0.7\textwidth]{./images/8_cubes_splitted_grid.png}
	\caption{-- Результат разбиения восьми шестигранников}
	\label{fig:6}
\end{figure}