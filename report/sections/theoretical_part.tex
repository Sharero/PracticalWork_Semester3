\section{Теоретическая часть}
\subsection{Алгоритм разбиения шестигранника}
В качестве шестигранника будет использоваться куб, представленный на рисунке \ref{fig:1}.
\begin{figure}[H]
	\centering
	\includegraphics[width=0.7\textwidth]{./images/computational_field.png}
	\caption{-- Исходная фигура}
	\label{fig:1}
\end{figure}
При разбиении нижняя грань остаётся неизменной, а все остальные грани могут делиться на 2 треугольника, в зависимости от проведённой диагонали. В данной работе будет использоваться разбиение, представленное на рисунке \ref{fig:2}.
\begin{figure}[H]
	\centering
	\includegraphics[width=0.7\textwidth]{./images/splitted_computational_field.png}
	\caption{-- Шаблон разбиения куба}
	\label{fig:2}
\end{figure}
При таком разбиении мы получаем 1 пирамиду и 4 тетраэдра с номерами узлов:
\par
Пирамида: $[0, 1, 2, 3, 4]$;
\par
Тетраэдр 1: $[1, 3, 4, 5]$;
\par
Тетраэдр 2: $[2, 3, 4, 6]$;
\par
Тетраэдр 3: $[3, 4, 5, 6]$;
\par
Тетраэдр 4: $[3, 5, 6, 7]$.
\par
\hspace{-30pt}\textbf{Входные данные:}
\par
global\_nodes\_coordinates --  координаты узлов всех шестигранников, которые учавствуют в разбиении.
\par
global\_nodes\_indicies -- номера узлов для каждого шестигранника, который учавствует в разбиении.
\par
\hspace{-30pt}\textbf{Выходные данные:}
\par
tetrahedron\_indicies --  вектор, содержащий номера узлов для каждого тетраэдра, полученного после разбиения.
\par
pyramide\_indicies -- вектор, содержащий номера узлов для каждой пирамиды, полученной после разбиения.
\par
\hspace{-30pt}Сам алгоритм разбиения заключается в следующем:
\begin{enumerate}
	\item \textbf{Сопоставление глобальных и локальных номеров для узлов куба.} Каждой вершине куба соответствует глобальный индекс из исходной сетки. Мы строим локальное упорядочение этих вершин относительно нижней фиксированной грани куба. В результате каждой вершине куба присваивается локальный номер от 0 до 7, где первые 4 номера -- это вершины нижней грани, а оставшиеся 4 -- это вершины верхней грани.
	\item \textbf{Применение шаблона разбиения.} 
	\item \textbf{Проверка ориентации тетраэдров.} После построения всех тетраэдров проверяется их ориентированный объём. Если объём отрицательный у тетраэдра меняются два локальных номера вершин.
	\item \textbf{Сохранение результатов.} Полученные номера вершин пирамиды и тетраэдров сохраняются в отдельные списки для последующей визуализации.
\end{enumerate}
\subsection{Визуализация сетки}
Для визуализации сетки на языке программирования C++ было принято решение использовать библиотеку VTK (Visualization Toolkit). Она позволяет работать с объёмными сетками, отображать их в 3D, настраивать цвета, прозрачность и границы ячеек, а также предоставляет интерактивное вращение и масштабирование объектов.
\par
Сам алгоритм визуализации разбиения куба на пирамиду и тетраэдры заключается в следующем:
\begin{enumerate}
	\item \textbf{Построение сетки для VTK.} Из списков глобальных координат вершин и индексов ячеек создаётся объект vtkUnstructuredGrid. Каждой ячейке присваивается тип: пирамидa (1) или тетраэдр (2).
	\item \textbf{Фильтрация элементов по типу.} Используются фильтры vtkThreshold, чтобы отдельно выбрать пирамиды и тетраэдры по их типу.
	\item \textbf{Создание поверхностных объектов.} Для каждой группы ячеек применяется фильтр vtkDataSetSurfaceFilter, который преобразует объёмные ячейки в поверхности для отображения.
	\item \textbf{Настройка отображения.} Создаются объекты vtkActor для каждой группы, задаются цвета, прозрачность и видимость границ.
	\item \textbf{Рендеринг и интерактивность.} Все акторы добавляются на vtkRenderer, который помещается в окно визуализации vtkRenderWindow. Затем запускается интерактивный рендерер для вращения, масштабирования и полного обзора разбиения куба.
\end{enumerate}