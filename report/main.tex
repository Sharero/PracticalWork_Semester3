\documentclass[a4paper, 14pt]{extarticle}

\usepackage{./styles/document}
\usepackage{./styles/code}
\usepackage{./styles/section}

\begin{document}
	\includepdf[pages=-]{./images/title.pdf}
	
	\tableofcontents
	\thispagestyle{empty}
	
	\section*{Введение}
\addcontentsline{toc}{section}{Введение}
Целью данной практической работы является написание программы, которая умеет разбивать сетку из шестигранников на пирамиду и тетраэдры, а так же умеющую визуализировать полученное разбиение.
\par
Результатом работы станет программа, визуализирующая разбиение куба на пирамиду с основанием на нижней грани куба и тетраэдрами в остальном объеме.
	\section{Теоретическая часть}
\subsection{Алгоритм разбиения шестигранника}
В качестве шестигранника будет использоваться куб, представленный на рисунке \ref{fig:1}.
\begin{figure}[H]
	\centering
	\includegraphics[width=0.7\textwidth]{./images/computational_field.png}
	\caption{-- Исходная фигура}
	\label{fig:1}
\end{figure}
При разбиении нижняя грань остаётся неизменной, а все остальные грани могут делиться на 2 треугольника, в зависимости от проведённой диагонали. В данной работе будет использоваться разбиение, представленное на рисунке \ref{fig:2}.
\begin{figure}[H]
	\centering
	\includegraphics[width=0.7\textwidth]{./images/splitted_computational_field.png}
	\caption{-- Шаблон разбиения куба}
	\label{fig:2}
\end{figure}
При таком разбиении мы получаем 1 пирамиду и 4 тетраэдра с номерами узлов:
\par
Пирамида: $[0, 1, 2, 3, 4]$;
\par
Тетраэдр 1: $[1, 3, 4, 5]$;
\par
Тетраэдр 2: $[2, 3, 4, 6]$;
\par
Тетраэдр 3: $[3, 4, 5, 6]$;
\par
Тетраэдр 4: $[3, 5, 6, 7]$.
\par
\hspace{-30pt}\textbf{Входные данные:}
\par
global\_nodes\_coordinates --  координаты узлов всех шестигранников, которые учавствуют в разбиении.
\par
global\_nodes\_indicies -- номера узлов для каждого шестигранника, который учавствует в разбиении.
\par
\hspace{-30pt}\textbf{Выходные данные:}
\par
tetrahedron\_indicies --  вектор, содержащий номера узлов для каждого тетраэдра, полученного после разбиения.
\par
pyramide\_indicies -- вектор, содержащий номера узлов для каждой пирамиды, полученной после разбиения.
\par
\hspace{-30pt}Сам алгоритм разбиения заключается в следующем:
\begin{enumerate}
	\item \textbf{Сопоставление глобальных и локальных номеров для узлов куба.} Каждой вершине куба соответствует глобальный индекс из исходной сетки. Мы строим локальное упорядочение этих вершин относительно нижней фиксированной грани куба. В результате каждой вершине куба присваивается локальный номер от 0 до 7, где первые 4 номера -- это вершины нижней грани, а оставшиеся 4 -- это вершины верхней грани.
	\item \textbf{Применение шаблона разбиения.} 
	\item \textbf{Проверка ориентации тетраэдров.} После построения всех тетраэдров проверяется их ориентированный объём. Если объём отрицательный у тетраэдра меняются два локальных номера вершин.
	\item \textbf{Сохранение результатов.} Полученные номера вершин пирамиды и тетраэдров сохраняются в отдельные списки для последующей визуализации.
\end{enumerate}
\subsection{Визуализация сетки}
Для визуализации сетки на языке программирования C++ было принято решение использовать библиотеку VTK (Visualization Toolkit). Она позволяет работать с объёмными сетками, отображать их в 3D, настраивать цвета, прозрачность и границы ячеек, а также предоставляет интерактивное вращение и масштабирование объектов.
\par
Сам алгоритм визуализации разбиения куба на пирамиду и тетраэдры заключается в следующем:
\begin{enumerate}
	\item \textbf{Построение сетки для VTK.} Из списков глобальных координат вершин и индексов ячеек создаётся объект vtkUnstructuredGrid. Каждой ячейке присваивается тип: пирамидa (1) или тетраэдр (2).
	\item \textbf{Фильтрация элементов по типу.} Используются фильтры vtkThreshold, чтобы отдельно выбрать пирамиды и тетраэдры по их типу.
	\item \textbf{Создание поверхностных объектов.} Для каждой группы ячеек применяется фильтр vtkDataSetSurfaceFilter, который преобразует объёмные ячейки в поверхности для отображения.
	\item \textbf{Настройка отображения.} Создаются объекты vtkActor для каждой группы, задаются цвета, прозрачность и видимость границ.
	\item \textbf{Рендеринг и интерактивность.} Все акторы добавляются на vtkRenderer, который помещается в окно визуализации vtkRenderWindow. Затем запускается интерактивный рендерер для вращения, масштабирования и полного обзора разбиения куба.
\end{enumerate}
	\section{Тестирование}
Тестирование проводилось на сетках, содержащих 1, 2, 4 и 8 шестигранников. Целью было убедиться в корректности алгоритма разбиения кубических элементов на пирамиды и тетраэдры, а также проверить визуализацию полученной сетки.

\subsection{Тестирование на 1 шестиграннике}
\hspace{-30pt}\textbf{Входные данные:}
\begin{itemize}
	\item \texttt{global\_node\_coordinates:}
	\begin{verbatim}
		0.0 0.0 0.0
		1.0 0.0 0.0
		0.0 1.0 0.0
		1.0 1.0 0.0
		0.0 0.0 1.0
		1.0 0.0 1.0
		0.0 1.0 1.0
		1.0 1.0 1.0
	\end{verbatim}
	\item \texttt{global\_nodes\_indicies:} 	
	\begin{verbatim}
		0 1 2 3 4 5 6 7
	\end{verbatim}
\end{itemize}

\vspace{-24pt}\hspace{-30pt}\textbf{Выходные данные:}
\begin{itemize}
	\item \textbf{Пирамида:} 0(0,0,0) 1(1,0,0) 3(1,1,0) 2(0,1,0) 4(0,0,1)
	\item \textbf{Тетраэдр 1:} 1(1,0,0) 3(1,1,0) 4(0,0,1) 5(1,0,1)
	\item \textbf{Тетраэдр 2:} 3(1,1,0) 2(0,1,0) 4(0,0,1) 6(0,1,1)
	\item \textbf{Тетраэдр 3:} 3(1,1,0) 4(0,0,1) 5(1,0,1) 6(0,1,1)
	\item \textbf{Тетраэдр 4:} 5(1,0,1) 3(1,1,0) 6(0,1,1) 7(1,1,1)
\end{itemize}

\hspace{-30pt}Результат разбиения представлены на рисунке \ref{fig:3}.

\begin{figure}[H]
	\centering
	\includegraphics[width=0.7\textwidth]{./images/1_cube_splitted_grid.png}
	\caption{-- Результат разбиения одного шестигранника}
	\label{fig:3}
\end{figure}

\subsection{Тестирование на 2 шестигранниках}
\hspace{-30pt}\textbf{Входные данные:}
\begin{itemize}
	\item \texttt{global\_node\_coordinates:}
	\begin{verbatim}
		0.0 0.0 0.0
		1.0 0.0 0.0
		2.0 0.0 0.0
		0.0 1.0 0.0
		1.0 1.0 0.0
		2.0 1.0 0.0
		0.0 0.0 1.0
		1.0 0.0 1.0
		2.0 0.0 1.0
		0.0 1.0 1.0
		1.0 1.0 1.0
		2.0 1.0 1.0
	\end{verbatim}
	\item \texttt{global\_nodes\_indicies:} 	
	\begin{verbatim}
		0 1 3 4 6 7 9 10
		1 2 4 5 7 8 10 11
	\end{verbatim}
\end{itemize}

\vspace{-24pt}\hspace{-30pt}\textbf{Выходные данные:}
\begin{itemize}
	\item \textbf{Пирамида 1:} 0(0,0,0) 1(1,0,0) 4(1,1,0) 3(0,1,0) 6(0,0,1) 
	\item \textbf{Пирамида 2:} 1(1,0,0) 2(2,0,0) 5(2,1,0) 4(1,1,0) 7(1,0,1) 
	\item \textbf{Тетраэдр 1:} 1(1,0,0) 4(1,1,0) 6(0,0,1) 7(1,0,1) 
	\item \textbf{Тетраэдр 2:} 4(1,1,0) 3(0,1,0) 6(0,0,1) 9(0,1,1) 
	\item \textbf{Тетраэдр 3:} 4(1,1,0) 6(0,0,1) 7(1,0,1) 9(0,1,1) 
	\item \textbf{Тетраэдр 4:} 7(1,0,1) 4(1,1,0) 9(0,1,1) 10(1,1,1) 
	\item \textbf{Тетраэдр 5:} 2(2,0,0) 5(2,1,0) 7(1,0,1) 8(2,0,1) 
	\item \textbf{Тетраэдр 6:} 5(2,1,0) 4(1,1,0) 7(1,0,1) 10(1,1,1) 
	\item \textbf{Тетраэдр 7:} 5(2,1,0) 7(1,0,1) 8(2,0,1) 10(1,1,1) 
	\item \textbf{Тетраэдр 8:} 8(2,0,1) 5(2,1,0) 10(1,1,1) 11(2,1,1)
\end{itemize}

\hspace{-30pt}Результат разбиения представлены на рисунке \ref{fig:4}.

\begin{figure}[H]
	\centering
	\includegraphics[width=0.7\textwidth]{./images/2_cubes_splitted_grid.png}
	\caption{-- Результат разбиения двух шестигранников}
	\label{fig:4}
\end{figure}

\subsection{Тестирование на 4 шестигранниках}
\hspace{-30pt}\textbf{Входные данные:}
\begin{itemize}
	\item \texttt{global\_node\_coordinates:}
	\begin{verbatim}
		0.0 0.0 0.0
		1.0 0.0 0.0
		2.0 0.0 0.0
		0.0 1.0 0.0
		1.0 1.0 0.0
		2.0 1.0 0.0
		0.0 2.0 0.0
		1.0 2.0 0.0
		2.0 2.0 0.0
		0.0 0.0 1.0
		1.0 0.0 1.0
		2.0 0.0 1.0
		0.0 1.0 1.0
		1.0 1.0 1.0
		2.0 1.0 1.0
		0.0 2.0 1.0
		1.0 2.0 1.0
		2.0 2.0 1.0
	\end{verbatim}
	\item \texttt{global\_nodes\_indicies:} 	
	\begin{verbatim}
		0 1 3 4 9 10 12 13
		1 2 4 5 10 11 13 14
		3 4 6 7 12 13 15 16
		4 5 7 8 13 14 16 17
	\end{verbatim}
\end{itemize}

\vspace{-24pt}\hspace{-30pt}\textbf{Выходные данные:}
\begin{itemize}
	\item \textbf{Пирамида 1:} 0(0,0,0) 1(1,0,0) 4(1,1,0) 3(0,1,0) 9(0,0,1) 
	\item \textbf{Пирамида 2:} 1(1,0,0) 2(2,0,0) 5(2,1,0) 4(1,1,0) 10(1,0,1) 
	\item \textbf{Пирамида 3:} 3(0,1,0) 4(1,1,0) 7(1,2,0) 6(0,2,0) 12(0,1,1) 
	\item \textbf{Пирамида 4:} 4(1,1,0) 5(2,1,0) 8(2,2,0) 7(1,2,0) 13(1,1,1) 
	\item \textbf{Тетраэдр 1:} 1(1,0,0) 4(1,1,0) 9(0,0,1) 10(1,0,1) 
	\item \textbf{Тетраэдр 2:} 4(1,1,0) 3(0,1,0) 9(0,0,1) 12(0,1,1) 
	\item \textbf{Тетраэдр 3:} 4(1,1,0) 9(0,0,1) 10(1,0,1) 12(0,1,1) 
	\item \textbf{Тетраэдр 4:} 10(1,0,1) 4(1,1,0) 12(0,1,1) 13(1,1,1) 
	\item \textbf{Тетраэдр 5:} 2(2,0,0) 5(2,1,0) 10(1,0,1) 11(2,0,1) 
	\item \textbf{Тетраэдр 6:} 5(2,1,0) 4(1,1,0) 10(1,0,1) 13(1,1,1) 
	\item \textbf{Тетраэдр 7:} 5(2,1,0) 10(1,0,1) 11(2,0,1) 13(1,1,1) 
	\item \textbf{Тетраэдр 8:} 11(2,0,1) 5(2,1,0) 13(1,1,1) 14(2,1,1) 
	\item \textbf{Тетраэдр 9:} 4(1,1,0) 7(1,2,0) 12(0,1,1) 13(1,1,1) 
	\item \textbf{Тетраэдр 10:} 7(1,2,0) 6(0,2,0) 12(0,1,1) 15(0,2,1) 
	\item \textbf{Тетраэдр 11:} 7(1,2,0) 12(0,1,1) 13(1,1,1) 15(0,2,1) 
	\item \textbf{Тетраэдр 12:} 13(1,1,1) 7(1,2,0) 15(0,2,1) 16(1,2,1) 
	\item \textbf{Тетраэдр 13:} 5(2,1,0) 8(2,2,0) 13(1,1,1) 14(2,1,1) 
	\item \textbf{Тетраэдр 14:} 8(2,2,0) 7(1,2,0) 13(1,1,1) 16(1,2,1) 
	\item \textbf{Тетраэдр 15:} 8(2,2,0) 13(1,1,1) 14(2,1,1) 16(1,2,1) 
	\item \textbf{Тетраэдр 16:} 14(2,1,1) 8(2,2,0) 16(1,2,1) 17(2,2,1)
\end{itemize}

\hspace{-30pt}Результат разбиения представлены на рисунке \ref{fig:5}.

\begin{figure}[H]
	\centering
	\includegraphics[width=0.7\textwidth]{./images/4_cubes_splitted_grid.png}
	\caption{-- Результат разбиения четырех шестигранников}
	\label{fig:5}
\end{figure}

\subsection{Тестирование на 4 шестигранниках}
\hspace{-30pt}\textbf{Входные данные:}
\begin{itemize}
	\item \texttt{global\_node\_coordinates:}
	\begin{verbatim}
		0.0 0.0 0.0
		1.0 0.0 0.0
		2.0 0.0 0.0
		0.0 1.0 0.0
		1.0 1.0 0.0
		2.0 1.0 0.0
		0.0 2.0 0.0
		1.0 2.0 0.0
		2.0 2.0 0.0
		0.0 0.0 1.0
		1.0 0.0 1.0
		2.0 0.0 1.0
		0.0 1.0 1.0
		1.0 1.0 1.0
		2.0 1.0 1.0
		0.0 2.0 1.0
		1.0 2.0 1.0
		2.0 2.0 1.0
		0.0 0.0 2.0
		1.0 0.0 2.0
		2.0 0.0 2.0
		0.0 1.0 2.0
		1.0 1.0 2.0
		2.0 1.0 2.0
		0.0 2.0 2.0
		1.0 2.0 2.0
		2.0 2.0 2.0
	\end{verbatim}
	\item \texttt{global\_nodes\_indicies:} 	
	\begin{verbatim}
		0 1 3 4 9 10 12 13
		1 2 4 5 10 11 13 14
		3 4 6 7 12 13 15 16
		4 5 7 8 13 14 16 17
		9 10 12 13 18 19 21 22
		10 11 13 14 19 20 22 23
		12 13 15 16 21 22 24 25
		13 14 16 17 22 23 25 26
	\end{verbatim}
\end{itemize}

\vspace{-24pt}\hspace{-30pt}\textbf{Выходные данные:}
\begin{itemize}
	\item \textbf{Пирамида 1:} 0(0,0,0) 1(1,0,0) 4(1,1,0) 3(0,1,0) 9(0,0,1) 
	\item \textbf{Пирамида 2:} 1(1,0,0) 2(2,0,0) 5(2,1,0) 4(1,1,0) 10(1,0,1) 
	\item \textbf{Пирамида 3:} 3(0,1,0) 4(1,1,0) 7(1,2,0) 6(0,2,0) 12(0,1,1) 
	\item \textbf{Пирамида 4:} 4(1,1,0) 5(2,1,0) 8(2,2,0) 7(1,2,0) 13(1,1,1) 
	\item \textbf{Пирамида 5:} 9(0,0,1) 10(1,0,1) 13(1,1,1) 12(0,1,1) 18(0,0,2) 
	\item \textbf{Пирамида 6:} 10(1,0,1) 11(2,0,1) 14(2,1,1) 13(1,1,1) 19(1,0,2) 
	\item \textbf{Пирамида 7:} 12(0,1,1) 13(1,1,1) 16(1,2,1) 15(0,2,1) 21(0,1,2) 
	\item \textbf{Пирамида 8:} 13(1,1,1) 14(2,1,1) 17(2,2,1) 16(1,2,1) 22(1,1,2) 
	\item \textbf{Тетраэдр 1:} 1(1,0,0) 4(1,1,0) 9(0,0,1) 10(1,0,1) 
	\item \textbf{Тетраэдр 2:} 4(1,1,0) 3(0,1,0) 9(0,0,1) 12(0,1,1) 
	\item \textbf{Тетраэдр 3:} 4(1,1,0) 9(0,0,1) 10(1,0,1) 12(0,1,1) 
	\item \textbf{Тетраэдр 4:} 10(1,0,1) 4(1,1,0) 12(0,1,1) 13(1,1,1) 
	\item \textbf{Тетраэдр 5:} 2(2,0,0) 5(2,1,0) 10(1,0,1) 11(2,0,1) 
	\item \textbf{Тетраэдр 6:} 5(2,1,0) 4(1,1,0) 10(1,0,1) 13(1,1,1) 
	\item \textbf{Тетраэдр 7:} 5(2,1,0) 10(1,0,1) 11(2,0,1) 13(1,1,1) 
	\item \textbf{Тетраэдр 8:} 11(2,0,1) 5(2,1,0) 13(1,1,1) 14(2,1,1) 
	\item \textbf{Тетраэдр 9:} 4(1,1,0) 7(1,2,0) 12(0,1,1) 13(1,1,1) 
	\item \textbf{Тетраэдр 10:} 7(1,2,0) 6(0,2,0) 12(0,1,1) 15(0,2,1) 
	\item \textbf{Тетраэдр 11:} 7(1,2,0) 12(0,1,1) 13(1,1,1) 15(0,2,1) 
	\item \textbf{Тетраэдр 12:} 13(1,1,1) 7(1,2,0) 15(0,2,1) 16(1,2,1) 
	\item \textbf{Тетраэдр 13:} 5(2,1,0) 8(2,2,0) 13(1,1,1) 14(2,1,1) 
	\item \textbf{Тетраэдр 14:} 8(2,2,0) 7(1,2,0) 13(1,1,1) 16(1,2,1) 
	\item \textbf{Тетраэдр 15:} 8(2,2,0) 13(1,1,1) 14(2,1,1) 16(1,2,1) 
	\item \textbf{Тетраэдр 16:} 14(2,1,1) 8(2,2,0) 16(1,2,1) 17(2,2,1) 
	\item \textbf{Тетраэдр 17:} 10(1,0,1) 13(1,1,1) 18(0,0,2) 19(1,0,2) 
	\item \textbf{Тетраэдр 18:} 13(1,1,1) 12(0,1,1) 18(0,0,2) 21(0,1,2) 
	\item \textbf{Тетраэдр 19:} 13(1,1,1) 18(0,0,2) 19(1,0,2) 21(0,1,2) 
	\item \textbf{Тетраэдр 20:} 19(1,0,2) 13(1,1,1) 21(0,1,2) 22(1,1,2) 
	\item \textbf{Тетраэдр 21:} 11(2,0,1) 14(2,1,1) 19(1,0,2) 20(2,0,2) 
	\item \textbf{Тетраэдр 22:} 14(2,1,1) 13(1,1,1) 19(1,0,2) 22(1,1,2) 
	\item \textbf{Тетраэдр 23:} 14(2,1,1) 19(1,0,2) 20(2,0,2) 22(1,1,2) 
	\item \textbf{Тетраэдр 24:} 20(2,0,2) 14(2,1,1) 22(1,1,2) 23(2,1,2) 
	\item \textbf{Тетраэдр 25:} 13(1,1,1) 16(1,2,1) 21(0,1,2) 22(1,1,2) 
	\item \textbf{Тетраэдр 26:} 16(1,2,1) 15(0,2,1) 21(0,1,2) 24(0,2,2) 
	\item \textbf{Тетраэдр 27:} 16(1,2,1) 21(0,1,2) 22(1,1,2) 24(0,2,2) 
	\item \textbf{Тетраэдр 28:} 22(1,1,2) 16(1,2,1) 24(0,2,2) 25(1,2,2) 
	\item \textbf{Тетраэдр 29:} 14(2,1,1) 17(2,2,1) 22(1,1,2) 23(2,1,2) 
	\item \textbf{Тетраэдр 30:} 17(2,2,1) 16(1,2,1) 22(1,1,2) 25(1,2,2) 
	\item \textbf{Тетраэдр 31:} 17(2,2,1) 22(1,1,2) 23(2,1,2) 25(1,2,2) 
	\item \textbf{Тетраэдр 32:} 23(2,1,2) 17(2,2,1) 25(1,2,2) 26(2,2,2)
\end{itemize}

\hspace{-30pt}Результат разбиения представлены на рисунке \ref{fig:6}.

\begin{figure}[H]
	\centering
	\includegraphics[width=0.7\textwidth]{./images/8_cubes_splitted_grid.png}
	\caption{-- Результат разбиения восьми шестигранников}
	\label{fig:6}
\end{figure}
	\section{Заключение}
В результате практической работы был изучен метод визуализации данных на С++ через VTK. Реализована программа, успешно разбиваюшая шестигранники на пирамиду и тетраэдры по заданному шаблону.
	
	\begin{thebibliography}{9}
		
		\bibitem{MKE}  Метод конечных элементов для решения скалярных и векторных задач : учеб. Пособие / Ю.Г. Соловейчик, М.Э. Рояк, М.Г. Персова -- Новосибирск: Изд-во НГТУ, 2007.-- 896 с. ("Учебники НГТУ").
		
		\bibitem{Зенкевич} Зенкевич О., Морган К. Конечные элементы и аппроксимация. -- М.: Мир, 1986.
		
		\bibitem{vtk}
		Schroeder W., Martin K., Lorensen B. The Visualization Toolkit: An Object-Oriented Approach to 3D Graphics. Edition 4.1. July 2018.
	\end{thebibliography}
	
	\appendix
	\section*{Приложение A. Текст программы}
\addcontentsline{toc}{section}{Приложение A. Текст программы}
\lstinputlisting[caption = {\texttt{main.cpp}}]{/home/yarik/NSTU/PracticalWork\_Semester3/src/main.cpp}
\lstinputlisting[caption = {\texttt{grid.h}}]{/home/yarik/NSTU/PracticalWork\_Semester3/include/grid.h}
\lstinputlisting[caption = {\texttt{math\_utils.h}}]{/home/yarik/NSTU/PracticalWork\_Semester3/include/math_utils.h}
\lstinputlisting[caption = {\texttt{visualize\_split.h}}]{/home/yarik/NSTU/PracticalWork\_Semester3/include/visualize_split.h}
\lstinputlisting[caption = {\texttt{grid.cpp}}]{/home/yarik/NSTU/PracticalWork\_Semester3/src/grid.cpp}
\lstinputlisting[caption = {\texttt{math\_utils.cpp}}]{/home/yarik/NSTU/PracticalWork\_Semester3/src/math_utils.cpp}
\lstinputlisting[caption = {\texttt{visualize\_split.cpp}}]{/home/yarik/NSTU/PracticalWork\_Semester3/src/visualize_split.cpp}
\end{document}